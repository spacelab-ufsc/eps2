%
% hardware.tex
%
% Copyright (C) 2021 by SpaceLab.
%
% EPS 2.0 Documentation
%
% This work is licensed under the Creative Commons Attribution-ShareAlike 4.0
% International License. To view a copy of this license,
% visit http://creativecommons.org/licenses/by-sa/4.0/.
%

%
% \brief Hardware project chapter.
%
% \author Gabriel Mariano Marcelino <gabriel.mm8@gmail.com>
% \author Yan Castro de Azeredo <yan.ufsceel@gmail.com>
%
% \institution Universidade Federal de Santa Catarina (UFSC)
%
% \version 0.1.1
%
% \date 2021/02/16
%

\chapter{Hardware} \label{ch:hardware}

The EPS2 is a 4 layer 1.6mm thick PCB with FR-4 dieletric. The module doesn't have any impedance control requirements, for this reason the layer stackup has 1oz (0.0347mm) thickness in inner and outer copper layers. In the following sections, the hardware design, interfaces, and standards are described in detail. Section are devided by subsystem blocks, following the diagrams present on \autoref{fig:mcu-block-diagram} and \autoref{fig:power-block-diagram}. The \autoref{fig:pcb-top}, \autoref{fig:pcb-bottom} and \autoref{fig:pcb-side} presents the 3D rendered images of the top, bottom and side views of the board, respectively.

\begin{figure}[!ht]
    \begin{center}
        \includegraphics[width=93mm]{figures/eps2-pcb-top.png}
        \caption{Top side of the PCB.}
        \label{fig:pcb-top}
    \end{center}
\end{figure}

\begin{figure}[!ht]
    \begin{center}
        \includegraphics[width=93mm]{figures/eps2-pcb-bottom.png}
        \caption{Bottom side of the PCB.}
        \label{fig:pcb-bottom}
    \end{center}
\end{figure}

\begin{figure}[!ht]
    \begin{center}
        \includegraphics[width=93mm]{figures/eps2-pcb-side.png}
        \caption{Side view of the PCB.}
        \label{fig:pcb-side}
    \end{center}
\end{figure}

\section{MCU}

The MCU\nomenclature{\textbf{MCU}}{\textit{Microcontroller.}} consists of a CPU, RAM Memory and Flash Memory (used for program storage and non-volatile status registers). The chosen MCU is a low power 16-bit RISC (\textit{MSP430F6659IPZR}) from Texas Instruments\cite{msp430f6659}. The \autoref{tab:msp430-summary} presents a summary of the main available features and \autoref{fig:msp430-diagram} shows the internal subsystems, descriptions, and peripherals. The microcontroller interfaces, configurations, and auxiliary components are described in the following topics.

\begin{table}[!h]
    \centering
    \begin{tabular}{cllllllll}
        \toprule[1.5pt]
        \textit{Flash} & \textit{SRAM} & \textit{Timers} & \textit{USCI} & \textit{ADC} & \textit{DAC} & \textit{GPIO} \\
        \midrule
        512KB  & 64KB  & 2  & 6 (SPI / I2C / UART)  & 12  & 2  & 74           \\
        \bottomrule[1.5pt]
    \end{tabular}
    \caption{Microcontroller features summary.}
    \label{tab:msp430-summary}
\end{table}

\begin{figure}[!ht]
    \begin{center}
        \includegraphics[width=\textwidth]{figures/msp430-diagram.png}
        \caption{Microcontroller internal diagram.}
        \label{fig:msp430-diagram}
    \end{center}
\end{figure}

\subsection{Interfaces Configuration}

The microcontroller has 6 Universal Serial Communication Interfaces (USCI) that can be configured to operate with different protocols and parameters. The \autoref{tab:usci-config} describes each interface configurations.
% Include interfaces section? "These interfaces are connected to different modules and peripherals, as presented in the \autoref{fig:diagram-interfaces}."

\begin{table}[!h]
    \centering
    \begin{tabular}{lrrrrl}
        \toprule[1.5pt]
        \textit{Interface} & \textit{Protocol (Index)} & \textit{Mode} & \textit{Word Length} & \textit{Data Rate} & \textit{Configuration} \\
        \midrule
        USCI\_A0           & UART0                     & -             & 8 bits               & 9600 bps           & Stop bits: 1 \\
                           &                           &               &                      &                    & Parity: None \\
        USCI\_A1           & SPI                       & Master        & 8 bits               & TBD                & Phase: High \\
                           &                           &               &                      &                    & Polarity: Low \\
        USCI\_A2           & UART1                     & -             & 8 bits               & 9600 bps           & Stop bits: 1 \\
                           &                           &               &                      &                    & Parity: None \\
        USCI\_B2           & I2C2                      & Slave         & 8 bits               & 100 kbps           & Address value: 0x36 \\
        \bottomrule[1.5pt]
    \end{tabular}
    \caption{USCI configuration.}
    \label{tab:usci-config}
\end{table}

\subsection{Voltage Reference}

To generate the 3 volts reference for the MCU internal ADC the EPS uses a \textit{595-REF5025AQDRQ1} chip.

\subsection{Clocks Configuration}

Besides the internal clock sources, the microcontroller has two dedicated clock inputs for external crystals: the main clock and the auxiliary clock inputs. There are a 32MHz (\textit{ABM8X-102-32.000MHZ-T}) and a 32.769kHz (\textit{ECS-.327-12.5-34S-TR}) crystals connected to these inputs, respectively. The first source is used for generating the Master Clock (MCLK) and the Subsystem Master Clock (SMCLK), which are used by the CPU and the internal peripheral modules. The second source is used for generating the Auxiliary Clock (ACLK) that handles the low-power modes and might be used for peripherals.


\subsection{Pinout}


\section{Batteries DaughterBoard}

Due to size restrictions the 4 cell batteries of the EPS2 were allocated to a dauhterboard named Battery Module 4 Cells, a.k.a BAT4C\nomenclature{\textbf{BAT4C}}{\textit{Battery Module 4 Cells.}}\cite{bat4c}. Both boards 3D models are assembled together in a EDA tool as seen in \autoref{fig:eps2-pcb-3d-battery}. BAT4C is connected below EPS2 in a board-to-board connector, the female counterpart (\textit{BAT4CIPS1-105-01-S-D}) present on the EPS is seen in \autoref{fig:battery-connector} with it pinout present on \autoref{tab:battery-connector}. For compability with the older version of the battery module the same connector pads are present near the middle section of the PCB. If the BAT4C is to be used, the connector for these pads must not be soldered, more detail on \autoref{ch:assembly}. Also external connectors are used for temperature measurement and control with RTDs and heaters, more details can be seen on \autoref{sec:rtd-picoblade} and \autoref{sec:heater-picoblade}. 

\begin{figure}[!ht]
    \begin{center}
        \includegraphics[width=93mm]{figures/eps2-pcb-3d-battery.png}
        \caption{EPS2 and BAT4C 3D models assembled.}
        \label{fig:eps2-pcb-3d-battery}
    \end{center}
\end{figure}

\begin{figure}[!ht]
    \begin{center}
        \includegraphics[width=0.5\textwidth]{figures/battery-connector.png}
        \caption{EPS2 battery connectors.}
        \label{fig:battery-connector}
    \end{center}
\end{figure}

\begin{table}[!h]
    \centering
    \begin{tabular}{cllll}
        \toprule[1.5pt]
        \textit{Pin} & \textit{Row} \\
        \midrule
        1            & $+$Vbat \\
        2            & $+$Vbat \\
        3            & $+$Vbat \\
        4            & $+$Vbat \\
        5            & $-$Vbat \\
        6            & $-$Vbat \\
        7            & $-$Vbat \\
        8            & $-$Vbat \\
        9            & Vbat\_Common \\
        10           & Vbat\_Common \\
        \bottomrule[1.5pt]
    \end{tabular}
    \caption{Battery connector pinout.}
    \label{tab:battery-connector}
\end{table}

\section{Solar Panels}

The energy harvesting system is based on solar energy conversion through ten solar panels attached to a 2U CubeSat structure. The solar panels are connected through six 4 pin PicoBlade connectors \textit{0533980471}. Because the EPS2 module has only six input connectors four pairs of solar panels will be connected in paralel. The connection scheme of the solar panels is visible in \autoref{fig:pcb-side}.

\begin{figure}[!ht]
    \begin{center}
        \includegraphics[width=0.75\textwidth]{figures/diagram-solar-panels.png}
        \caption{Solar panels connection to EPS2.}
        \label{fig:diagram-solar-panels}
    \end{center}
\end{figure}

\section{MPPT Subsystem}

On the MPPT subsystem the main components are the MPPT boost converters, solar panels voltage and current sensors. These measurement circuits are used to generate a voltage proportional to the variable being measured, in a range accepted by the MCU internal ADC\nomenclature{\textbf{ADC}}{\textit{Analog to Digital Converter.}}.

\subsection{MPPT Boost Converters}

There are three boost converters in the system, one for each couple of solar panels in parallel connection. Each one is a discrete boost with a \textit{HC9-220-R} inductor, a \textit{SI4166DY} mosfet as the switch and a \textit{B340LA-13-F} diode. There are six \textit{GRM32ER1E226KE15L} capacitors and two \textit{GRM216R71H103KA01D} capacitors connected in parallel in the boost output. The output filter is the same for all the converters as their outputs are tied together. The control PWM\nomenclature{\textbf{PWM}}{\textit{Pulse Width Modulation.}} signals are generated by the MCU at a frequency of nearly 500 kHz.

\subsection{Solar Panels Current}

The main component of the solar panels currents measurement circuit is the \textit{MAX9934TAUA+} current sense amplifier. It generates an output current proportional to the differential input voltage. The gain is 25 $\mu$A/mV. To make the measurements possible, the current goes through 50 m$\Omega$, 0.5 \% resistors, connected to the inputs of the amplifier, and the outputs are connected to 3.3 k$\Omega$ resistors. The output voltage of the circuit is given by:

\begin{equation}
V_{out} = I_{sense} \cdot R_{sense} \cdot G \cdot R_{out}
\end{equation}

\subsection{Solar Panels Voltage}

The solar panels voltage measurement circuit is composed by a voltage divider and an op-amp in a buffer configuration. The voltage divider is composed of a 93.1 k$\Omega$ resistor and an 100 k$\Omega$ resistor. The op-amp is a \textit{TLV341AIDBVR} chip. The output voltage is given by:

\begin{equation}
V_{out} = V_{sp} \cdot \frac{R_{2}}{R_{1} + R_{2}}
\end{equation}

\section{Batteries Managment Subsystem}

On the batteries managment subsystem the main components are the battery control circuit, external ADC chip, solar panels and batteries kill-switches, heater drivers and voltage sensors for the boosters output and main power bus. 

\subsection{Boost Converters Output Voltage}

The boost converters output voltage measurement circuit is very similar to the solar panels voltages measurement circuit, with the exception that the voltage divider is composed by a 300 k$\Omega$ resistor and an 100 k$\Omega$ resistor.

\subsection{Kill-Switches}

These switches are used to separate the solar panels and the batteries from the load during pre-flight and launch. Each one is composed of two \textit{SI4403-CDY-T1-GE3} P-channel mosfets in parallel, as a redundancy. When either the RBF\nomenclature{\textbf{RBF}}{\textit{Remove Before Flight.}} is in place or the kill-switches are pressed, the mosfets disconnect the loads from the sources.

\subsection{Remove Before Flight}

\subsection{Battery Control Circuit}

The batteries are monitored by the \textit{DS2775} chip. It measures several parameters and sends them to the EPS2 MCU via one-wire protocol. Also it automatically protects the batteries against short-circuits, overvoltage and undervoltage situations by switching two \textit{FDS6898AZ} mosfets.

\subsection{Main Power Bus Voltage}

The main power bus voltage measurement circuit is identical to the boost converters output voltage measurement circuit.

\subsection{ADC}

The \textit{ADS1248} chip generates a precise reference current to the RTDs, and samples the voltage proportional to the temperature established over the sensors. This voltage is converted to digital data and sent to the MCU via SPI protocol.

\subsection{Heaters Drivers}

The drivers are chopper converters controlled by the MCU, with a PWM frequency of 50 kHz. The switches of the chopper converters are \textit{Si4010DY} mosfets.

\section{Power Converters Subsystem}

The EPS2 has 6 integrated buck DC-DC regulators, all these are powered from the main power bus. Some regulators are always enabled, others are can be enabled or disabled by the EPS2 or other module.

\subsection{EPS/TTC Regulator}

To supply the TTC MCU (also called "Beacon MCU") and EPS2 MCU and its subcircuits a \textit{TPS5420QDRQ1} regulator is used, with and output voltage of 3.3 V and 2 A current capability. This regulator is always on.

There is a current measurement at the output of the EPS/TTC regulator. It also uses a \textit{MAX9934TAUA+} current sense amplifier, but with a shunt resistor of 75 m$\Omega$, 0.5 \% and the output connected to a 4.02 k$\Omega$ resistor.

\subsection{OBDH Regulator}

The OBDH is powered by a \textit{TPS5410QDRQ1} regulator, with an output voltage of 3.3 V and 1 A current capability. The EPS2 can enable/disable this regulator.

\subsection{Antenna Deployer Regulator}

The antenna deployment system has a dedicated regulator \textit{TPS5420QDRQ1}, with 3.3 V output voltage and 2 A current capability. This regulator is always on.

\subsection{Main Radio Transceiver Regulator}

The main radio transceiver responsible for the Downlink/Uplink of the CubeSat is powered by a \textit{TPS54540QDDARQ1} regulator, with an output voltage of 5V and 5A campability. The OBDH can enable/disable this regulator.

\subsection{Beacon Transceiver Regulator}

The Beacon transceiver is powered by a regulator \textit{TPS54540QDDARQ1} regulator, with 6V output voltage and 5A campability. The Beacon MCU can enable/disable this regulator.

\subsection{Payloads Regulator}

To power the payloads a \textit{TPS5430QDDARQ1} regulator is used. It has an output voltage of 5 V and 3 A current capability. The EPS2 can enable/disable this regulator.

\section{External Connectors}

The EPS2 module is connected to the other modules using the PC104 bus. The solar panels, the kill-switches, the remove before flight, the RTDs, the heater, the batteries charger connector and the JTAG pins are connected using Molex PicoBlade connectors. The EPS2 module also has a jumper that connects the MCU VCC to the JTAG VCC and a header to debug the board via UART protocol. In the following sections each connector is detailed, with a picture showing the location on the EPS2 PCB and a table explaining each pin function.

\subsection{PC104}

\begin{figure}[!ht]
    \begin{center}
        \includegraphics[width=0.5\textwidth]{figures/pc104-diagram}
        \label{fig:pc104-diagram}
        \caption{Reference diagram of the PC-104 bus.}
    \end{center}
\end{figure}

\begin{table}[!h]
    \centering
    \begin{tabular}{cllll}
        \toprule[1.5pt]
        \textit{Pin [A-B]} & \textit{H1A}     & \textit{H1B}     & \textit{H2A}  & \textit{H2B}  \\
        \midrule
        1-2                & -                & -                & -             & -             \\
        3-4                & -                & -                & -             & -             \\
        5-6                & -                & -                & UART\_RX      & -             \\
        7-8                & -                & -                & UART\_TX      & -             \\
        9-10               & -                & EN\_PWR\_5       & -             & -             \\
        11-12              & -                & EN\_PWR\_6       & -             & -             \\
        13-14              & -                & -                & -             & -             \\
        15-16              & -                & -                & -             & -             \\
        17-18              & -                & -                & -             & -             \\
        19-20              & -                & -                & -             & -             \\
        21-22              & -                & -                & -             & -             \\
        23-24              & -                & -                & -             & -             \\
        25-26              & -                & -                & PWR\_4\_5V    & PWR\_4\_5V    \\
        27-28              & -                & -                & PWR\_7\_3V3   & PWR\_7\_3V3   \\
        29-30              & GND              & GND              & GND           & GND           \\
        31-32              & GND              & GND              & GND           & GND           \\
        33-34              & -                & -                & -             & -             \\
        35-36              & -                & -                & PWR\_1\_3V3   & PWR\_1\_3V3   \\
        37-38              & -                & -                & -             & -             \\
        39-40              & -                & -                & -             & -             \\
        41-42              & -                & -                & -             & -             \\
        43-44              & -                & -                & -             & -             \\
        45-46              & PWR\_2\_3V3      & PWR\_2\_3V3      & PWR\_3\_BAT   & PWR\_3\_BAT   \\
        47-48              & PWR\_4\_5V       & PWR\_4\_5V       & -             & -             \\
        49-50              & PWR\_5\_5V       & PWR\_5\_5V       & I2C\_SDA      & -             \\
        51-52              & PWR\_6\_6V       & PWR\_6\_6V       & I2C\_SCL      & -             \\
        \bottomrule[1.5pt]
    \end{tabular}
    \caption{PC-104 connector pinout.}
    \label{tab:pc104-pins}
\end{table}

\begin{table}[!h]
    \centering
    \begin{tabular}{lL{0.29\textwidth}L{0.47\textwidth}}
        \toprule[1.5pt]
        \textbf{Signal}  & \textbf{Pin(s)}            & \textbf{Description} \\
        \midrule
        GND              & H1-29, H1-30, H1-31, H1-32, H2-29, H2-30, H2-31, H2-32 & Ground reference \\
        PWR\_1\_3V3      & H2-35, H2-36               & Power bus 1, 3.3 V, 2 A max. \\
        PWR\_2\_3V3      & H1-45, H1-46               & Power bus 2, 3.3 V, 1 A max. \\
        PWR\_3\_BAT      & H2-45, H2-46               & Power bus 3, battery terminals (+) \\
        PWR\_4\_5V       & H1-47, H1-48, H2-25, H2-26 & Power bus 4, 5 V, 3 A max. \\
        PWR\_5\_5V       & H1-49, H1-50               & Power bus 5, 5 V, 5 A max. \\
        PWR\_6\_6V       & H1-51, H1-52               & Power bus 6, 6 V, 5 A max. \\
        PWR\_7\_3V3      & H2-27, H2-28               & Power bus 7, 3.3 V, 2 A max. \\
        I2C\_SDA         & H2-49                      & Primary communication bus (data signal) \\
        I2C\_SCL         & H2-51                      & Primary communication bus (clock signal) \\
        UART\_RX         & H2-5                       & Secondary communication bus (RX) \\
        UART\_TX         & H2-7                       & Secondary communication bus (TX) \\
        EN\_PWR\_5       & H1-10                      & Enable signal of the power bus 5 \\
        EN\_PWR\_6       & H1-12                      & Enable signal of the power bus 6 \\
        \bottomrule[1.5pt]
    \end{tabular}
    \caption{PC-104 bus signal description.}
    \label{tab:pc104-signals}
\end{table}

\subsection{Solar Panels PicoBlades}


\subsection{Kill-Switches PicoBlades}


\subsection{RBF PicoBlade}


\subsection{RTDs PicoBlade} \label{sec:rtd-picoblade}


\subsection{Heater PicoBlade} \label{sec:heater-picoblade}


\subsection{External Batteries Charger PicoBlade}


\subsection{JTAG PicoBlade}


\subsection{Debug UART Pin Header}
