%
% system_overview.tex
%
% Copyright The EPS 2.0 Contributors.
%
% EPS 2.0 Documentation
%
% This work is licensed under the Creative Commons Attribution-ShareAlike 4.0
% International License. To view a copy of this license,
% visit http://creativecommons.org/licenses/by-sa/4.0/.
%

%
% \brief System overview chapter.
%
% \author Gabriel Mariano Marcelino <gabriel.mm8@gmail.com>
% \author Yan Castro de Azeredo <yan.ufsceel@gmail.com>
%
% \version 0.3.0
%
% \date 2021/02/10
%

\chapter{System Overview} \label{ch:system-overview}

The board has a MSP430 low-power MCU\nomenclature{\textbf{MCU}}{\textit{Microcontroller.}} that runs the firmware application intended to control and communicate with its peripherals, subsystems, and other modules. The programming language used is C, and the firmware was developed using the Code Composer Studio IDE\nomenclature{\textbf{IDE}}{\textit{Integrated Development Environment.}} (a.k.a. CCS) for compiling, programming and testing. The module has many tasks, such as interfacing internal peripherals and communicating with other boards over distinct protocols and time requirements. 
So, in order to improve predictability, a Real-Time Operating System (RTOS\nomenclature{\textbf{RTOS}}{\textit{Real Time Operating System.}}) is used to ensure that the deadlines are observed, even under a fault situation in a routine. The RTOS chosen is the FreeRTOS (v10.2.1) since it is designed for embedded systems applications and was already validated in space applications. The firmware architecture follows an abstraction layer scheme to facilitate higher level implementations and allow more portability across different hardware platforms; see \autoref{sec:system-layers} for more details.

The EPS 2.0 is compatible with GOMspace Solar Panels or with panels of similar characteristics. Algorithms are implemented to improve power generation through MPPT. Also, the load output can be regulated through measurements to achieve a more efficient power distribution to the nanosatellite.

\section{Product tree}

The product tree of the EPS 2.0 module is available in \autoref{fig:product-tree}.

\begin{figure}[!ht]
    \begin{center}
        \includegraphics[width=0.8\textwidth]{figures/product-tree.pdf}
        \caption{Product tree of the EPS 2.0 module.}
        \label{fig:product-tree}
    \end{center}
\end{figure}

\section{MCU Block Diagram}

The \autoref{fig:mcu-block-diagram} presents a simplified view of the module subsystems and interfaces through the microcontroller perspective. 
The MCU has a programming JTAG\nomenclature{\textbf{JTAG}}{\textit{Joint Test Action Group.}}, a dedicated UART\nomenclature{\textbf{UART}}{\textit{Universal Asynchronous Receiver/Transmitter.}} debug interface and 4 communication buses, divided in 4 different protocols (I2C\nomenclature{\textbf{I2C}}{\textit{Inter-Integrated Circuit.}}, SPI\nomenclature{\textbf{SPI}}{\textit{Serial Peripheral Interface.}} and UART). 

There is an I2C buffer to allow secure and proper communication with the OBDH 2.0 module \cite{obdh2}.
The SPI protocol is used for controlling and retrieving data from an additional ADC\nomenclature{\textbf{ADC}}{\textit{Analog-to-Digital Converter.}} IC that measures temperature sensors (RTDs\nomenclature{\textbf{RTD}}{\textit{Resistance Temperature Detector.}}) on the batteries board and solar panels.
Several parameters from the Batteries Management Subsystem are sent to the EPS 2.0 MCU via I2C protocol.
The UART bus that goes to the PC/104 is used for basic telemetry to be sent to the beacon microcontroller within the TTC module.
Besides these channels, there are GPIO\nomenclature{\textbf{GPIO}}{\textit{General Purpose Input/Output.}} connections for enabling and disabling power buses, for hard code PCB versioning, and some optional GPIOs that can be added and used though the PC/104 interface. 

The MCU reads the measurements of current and voltage of the solar panels from its ADC ports for the MPPT Subsystem; also, from these data, the MPPT is controlled by the microcontroller through PWM\nomenclature{\textbf{PWM}}{\textit{Pulse Width Modulation.}} signals.

An external charger is used to charge the batteries and kill-switches to power off the EPS 2.0 module during the test phase. Before launch, the kill-switches are also connected to the button switches present on a CubeSat structure.   

\begin{figure}[!ht]
    \begin{center}
        \includegraphics[width=0.75\textwidth]{figures/eps2_mcu_diagram.pdf}
        \caption{EPS 2.0 MCU Block diagram.}
        \label{fig:mcu-block-diagram}
    \end{center}
\end{figure}

\section{Power Block Diagram}

The \autoref{fig:power-block-diagram} presents a more detailed view of the power subsystems that complements the MCU Block Diagram. More details and descriptions about these hardware components and interfaces are provided in the \autoref{ch:hardware}.

\begin{figure}[!ht]
    \begin{center}
        \includegraphics[width=\textwidth]{figures/eps2_power_diagram.pdf}
        \caption{EPS 2.0 Power Block diagram.}
        \label{fig:power-block-diagram}
    \end{center}
\end{figure}

\section{System Layers} \label{sec:system-layers}

The system is divided into various abstraction layers to favor high-level firmware implementations. The \autoref{fig:system-layers} shows this scheme, composed of third-party drivers at the lowest layer above the hardware, the operating system as the base building block of the module, the devices handling implementation, and the application tasks in the highest layer. More details are provided in the \autoref{ch:firmware}.

\begin{figure}[!ht]
    \begin{center}
        \includegraphics[width=0.75\textwidth]{figures/eps-system-layers.pdf}
        \caption{System layers.}
        \label{fig:system-layers}
    \end{center}
\end{figure}

\section{Operation} \label{sec:operation}

The system operates through the sequential execution of routines (tasks in the context of the operating system) that are scheduled and multiplexed through time. Each routine has a priority and a periodicity, determining the subsequent execution, the set of functionalities currently running, and the memory usage management.

Besides this deterministic scheduling system, the routines have communication channels with each other through the usage of queues, which provides a robust synchronization scheme. The system operation and internal nuances are described in detail in the \autoref{ch:firmware}. This section is a top view, user perspective, to describe the module's operation.

\subsection{Execution Flow}

To better understand how the EPS2 will operate during the mission, a simplified representation of its operation is presented in \autoref{fig:execution-flow}. Overall, the EPS2 should control the temperature of the battery and its energy harvesting system operation and obtain the sensors' measurements periodically. The OBDH or TTC modules can communicate with the EPS2, which should receive or transmit data as requested. Furthermore, every 10 hours, the EPS2 will reset.

\begin{figure}[!ht]
    \begin{center}
        \includegraphics[width=0.5\textwidth]{figures/execution_flow.pdf}
        \caption{Simplified representation of the EPS2'S execution flow.}
        \label{fig:execution-flow}
    \end{center}
\end{figure}



%\subsection{Data Flow}
% Add here a flowchart or diagram showing where and how the data is generated and transferred across different modules and peripherals.

\subsection{Status LEDs} \label{status-leds}

On the development version of the board, there are ten LED\nomenclature{\textbf{LED}}{\textit{Light-Emitting Diode.}}s that indicate some behaviors of the systems. This set of LEDs can be seen on \autoref{fig:status-leds}.

\begin{figure}[!ht]
    \begin{center}
        \includegraphics[width=\textwidth]{figures/status_leds.png}
        \caption{Available status LEDs.}
        \label{fig:status-leds}
    \end{center}
\end{figure}

A description of each of these LEDs is available below:

\begin{itemize}
    \item \textbf{6V\_RADIO\_1\_LED}: Indicates that the radio 1 transceiver 6V power is being sourced.
    \item \textbf{5V\_RADIO\_0\_LED}: Indicates that the radio 0 transceiver 5V power is being sourced.
    \item \textbf{SYSTEM\_ON}: Heartbeat of the system. It blinks at a frequency of 1 Hz when the system is running properly.
    \item \textbf{SYSTEM\_FAULT}: Indicates an error during the board's running firmware.
    \item \textbf{3V3\_ANT}: Indicates that the antenna deployer 3.3V power is being sourced.
    \item \textbf{3V3\_MCU}: Indicates that the EPS2 MCU 3.3V power is being sourced.
    \item \textbf{3V3\_OBDH}: Indicates that the OBDH module 3.3V power is being sourced.
    \item \textbf{3V3\_EPS\_BEACON}: Indicates that the EPS2 board and beacon MCU 3.3V power is being sourced.
    \item \textbf{5V\_PAYLOADS\_LEDS}: Indicates that the payloads 5V power is being sourced.
    \item \textbf{VBUS\_LED}: Indicates that the main power bus from the batteries is being sourced.
\end{itemize}

These LEDs are not mounted in the flight version of the module.

\section{Hard Code Versioning}

On the EPS2 board, there are 2 GPIOs dedicated to hard code versioning.
The on-board firmware can read these pins to identify the correct version of the hardware project.
Each line can be either pulled to VCC or ground, representing in binary as 1 and 0, respectively.
The \autoref{tab:versioning-resistors} shows the versioning representation up to the project's latest revision.

\begin{table}[!h]
    \centering
    \begin{tabular}{lcc}
        \toprule[1.5pt]
        \textbf{Version}    &   \textbf{P3.5 (pin code) / 47 (pin number)}    &    \textbf{P3.4 (pin code) / 46 (pin number)}\\
        \midrule
        v0.1                & 0                 & 0              \\
        v0.2                & 0                 & 1              \\
        \bottomrule[1.5pt]
    \end{tabular}
    \caption{Hard code versioning table.}
    \label{tab:versioning-resistors}
\end{table}
